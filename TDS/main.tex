\documentclass[
	% -- opções da classe memoir --
	article,			% indica que é um artigo acadêmico
	12pt,				% tamanho da fonte
	oneside,			% para impressão apenas no recto. Oposto a twoside
	a4paper,			% tamanho do papel. 
	% -- opções da classe abntex2 --
	%chapter=TITLE,		% títulos de capítulos convertidos em letras maiúsculas
	%section=TITLE,		% títulos de seções convertidos em letras maiúsculas
	%subsection=TITLE,	% títulos de subseções convertidos em letras maiúsculas
	%subsubsection=TITLE % títulos de subsubseções convertidos em letras maiúsculas
	% -- opções do pacote babel --
	english,			% idioma adicional para hifenização
	brazil,				% o último idioma é o principal do documento
	sumario=tradicional
	]{abntex2}


% ---
% PACOTES
% ---

% ---
% Pacotes fundamentais 
% ---
\usepackage{sbc-template}
\usepackage{times}			% Usa a fonte Times new Roman
\usepackage[T1]{fontenc}		% Selecao de codigos de fonte.
\usepackage[utf8]{inputenc}		% Codificacao do documento (conversão automática dos acentos)
\usepackage{indentfirst}		% Indenta o primeiro parágrafo de cada seção.
\usepackage{nomencl} 			% Lista de simbolos
\usepackage{color}				% Controle das cores
\usepackage{graphicx}			% Inclusão de gráficos
\usepackage{microtype} 			% para melhorias de justificação

% ---
		
% ---
% Pacotes adicionais, usados apenas no âmbito do Modelo Canônico do abnteX2
% ---
\usepackage{lipsum}				% para geração de dummy text
% ---
		
% ---
% Pacotes de citações
% ---
\usepackage[brazilian,hyperpageref]{backref}	 % Paginas com as citações na bibl
\usepackage[alf]{abntex2cite}	% Citações padrão ABNT
% ---

% ---
% Configurações do pacote backref
% Usado sem a opção hyperpageref de backref
\renewcommand{\backrefpagesname}{Citado na(s) página(s):~}
% Texto padrão antes do número das páginas
\renewcommand{\backref}{}
% Define os textos da citação
\renewcommand*{\backrefalt}[4]{
	\ifcase #1 %
		Nenhuma citação no texto.%
	\or
		Citado na página #2.%
	\else
		%Citado #1 vezes nas páginas #2.%
        Citado nas páginas #2.%
	\fi}%
% ---

% --- Informações de dados para CAPA e FOLHA DE ROSTO ---

\title{Modelo de documento para as disciplas\\ TDS e PDS}

\author{Daniela dos Santos Santana\inst{1}, Gustavo Fortunato Puga\inst{2}, Johnata Souza Santicioli\inst{1},\\ Leonardo Andrade Motta de Lima\inst{3} }


\address{Curso Informática -- Instituto Federal de São Paulo
  (IFSP)\\
  }


% ---
% Configurações de aparência do PDF final

% alterando o aspecto da cor azul
\definecolor{blue}{RGB}{41,5,195}

% informações do PDF
\makeatletter
\hypersetup{
     	%pagebackref=true,
		pdftitle={\@title}, 
		pdfauthor={\@author},
    	pdfsubject={Modelo de artigo científico com abnTeX2},
	    pdfcreator={LaTeX with abnTeX2},
		pdfkeywords={abnt}{latex}{abntex}{abntex2}{atigo científico}, 
		colorlinks=true,       		% false: boxed links; true: colored links
    	linkcolor=blue,          	% color of internal links
    	citecolor=blue,        		% color of links to bibliography
    	filecolor=magenta,      		% color of file links
		urlcolor=blue,
		bookmarksdepth=4
}
\makeatother
% --- 

% ---
% compila o indice
% ---
\makeindex
% ---

% ---
% Altera as margens padrões
% ---
\setlrmarginsandblock{3cm}{3cm}{*} %{ESQUERDA}{DIREITA}
\setulmarginsandblock{3.5cm}{2.5cm}{*} %{SUPERIOR}{INFERIOR}
\checkandfixthelayout
% ---

% --- 
% Espaçamentos entre linhas e parágrafos 
% --- 

% O tamanho do parágrafo é dado por (está definido dentro do sbc-template.sty):
%\setlength{\parindent}{1.3cm}

% Controle do espaçamento entre um parágrafo e outro (está definido dentro do sbc-template.sty):
%\setlength{\parskip}{0.2cm}  % tente também \onelineskip

% Espaçamento simples
\SingleSpacing


% ----
% Início do documento
% ----
\begin{document}

% Seleciona o idioma do documento (conforme pacotes do babel)
%\selectlanguage{english}
\selectlanguage{brazil}


% Retira espaço extra obsoleto entre as frases.
\frenchspacing 

% ----------------------------------------------------------
% ELEMENTOS PRÉ-TEXTUAIS
% ----------------------------------------------------------

%---
%
% Se desejar escrever o artigo em duas colunas, descomente a linha abaixo
% e a linha com o texto ``FIM DE ARTIGO EM DUAS COLUNAS''.
% \twocolumn[    		% INICIO DE ARTIGO EM DUAS COLUNAS
%
%---

% página de titulo principal (obrigatório)
\maketitle

% titulo em outro idioma (opcional)

\begin{abstract}
  This meta-paper describes the style to be used in articles and short papers
  for SBC conferences. For papers in English, you should add just an abstract
  while for the papers in Portuguese, we also ask for an abstract in
  Portuguese (``resumo''). In both cases, abstracts should not have more than
  10 lines and must be in the first page of the paper.
\end{abstract}
     
\begin{resumo1} 
  Este meta-artigo descreve o estilo a ser usado na confecção de artigos e
  resumos de artigos para publicação nos anais das conferências organizadas
  pela SBC. É solicitada a escrita de resumo e abstract apenas para os artigos
  escritos em português. Artigos em inglês deverão apresentar apenas abstract.
  Nos dois casos, o autor deve tomar cuidado para que o resumo (e o abstract)
  não ultrapassem 10 linhas cada, sendo que ambos devem estar na primeira
  página do artigo.
\end{resumo1}



% ]  				% FIM DE ARTIGO EM DUAS COLUNAS
% ---

%\begin{center}\smaller
%\textbf{Data de submissão e aprovação}: elemento obrigatório. Indicar dia, mês e ano

%\textbf{Identificação e disponibilidade}: elemento opcional. Pode ser indicado o endereço eletrônico, DOI, suportes e outras informações relativas ao acesso.
%\end{center}

% ----------------------------------------------------------
% ELEMENTOS TEXTUAIS
% ----------------------------------------------------------
\textual

% ----------------------------------------------------------
% Introdução
% ----------------------------------------------------------
\section{Introdução}
O objetivo deste documento é expor as regras e formatos para o desenvolvimento dos documentos das disciplinas de TDS e PDS. Este utiliza como base tanto o modelo da SBC\footnote{\url{https://www.overleaf.com/latex/templates/sbc-conferences-template/blbxwjwzdngr}} quanto o fornecido pela abnTeX\footnote{\url{http://www.latex-project.org/lppl.txt}}.

O documento final deve ter \textbf{entre 12 e 15 páginas}, descontados na contagem apenas os anexos e apêndices. Cada professor é livre para definir junto aos alunos o que deve estar em cada uma das duas seções supracitadas.

As seções aqui colocadas são apenas EXEMPLOS do que deve estar contido, podendo ser alteradas em conjunto entre orientadores e alunos, assim como as devidas explicações dentro delas.

Atenção especial para a quantidade de texto colocada em cada uma das seções. De forma geral, não faz muito sentido colocar uma seção para escrever apenas um parágrafo ou 6, 7 linhas.

Para dicas e como fazer tabelas, listas, figuras, citações e etc, bem como evitar erros em projetos e no próprio documento, consultar (entre outros) os capítulos 6, 7, 8 e 9 do documento: \url{https://www.overleaf.com/project/58a3a66af9bb74023ba1bd56}.


Ao preparar a Introdução é importante estar atento a:
    \begin{citacao}
    "A introdução é a parte responsável pela apresentação do trabalho, desde a delimitação do tema até a forma como está organizado. Sugere-se que seja feita a introdução do assunto, de modo a: discorrer sobre o tema do trabalho; apresentação do problema e objetivos da pesquisa; exposição da justificativa; finalmente, informar em quantos capítulos o texto foi dividido e apresentar os principais elementos que os compõem . Por apresentar estes elementos, a introdução requer uma revisão detalhada ao final do trabalho para manter a coesão textual."  \cite{NormalizacaoIFSP}.
    \end{citacao}

\subsection{Objetivos}

Os objetivos precisam seguir uma linha de raciocínio ao serem traçados.
    \begin{citacao}
    "O objetivo normalmente comporta uma hipótese de trabalho. Um bom objetivo de pesquisa normalmente terá
    a forma “demonstrar que a hipótese x é verdadeira”.
    Nem todo objetivo pode ser considerado um bom objetivo de pesquisa. Por exemplo, algo do tipo “o objetivo
    deste trabalho é aumentar os meus conhecimentos na área de estudo” pode até ser muito sincero, mas não
    convence ninguém de que algum conhecimento novo para a humanidade será produzido. Portanto, isso deve ser evitado.
    Outro objetivo algumas vezes encontrado é a forma “propor…”. Alguma coisa é proposta, normalmente um
    método, uma abordagem, uma técnica, um algoritmo, uma comparação, ou qualquer outra coisa. A questão é:
    se o autor fizer a proposta, então o objetivo estará atingido? Se o aluno se propõe a propor e propôs, então está
    proposto! O que for proposto não é necessariamente melhor ou diferente daquilo que já existia antes. Então, o
    estágio da pesquisa neste caso ainda é dos mais ingênuos.
    É necessário que o objetivo diga que aquilo que está sendo proposto é melhor do que alguma outra coisa ou
    que resolve algum problema que antes não podia ser resolvido."  \cite{PESQUISA:RAUL}.
    \end{citacao}

Os objetivos PODEM ser divididos em: Principal e Específicos (ou Secundários).

Em relação ao objetivo principal:

    \begin{citacao}
    "Atenção especial é dada ao verbo que apresenta o objetivo.
    Analisa-se também se o objetivo apresentado define claramente uma pesquisa científica ou um objetivo
    tecnológico, como, por exemplo, a implementação de um sistema."
    \lipsum[5] \cite{PESQUISA:RAUL}.
    \end{citacao}

Em relação aos objetivos específicos (ou Secundários):

    \begin{citacao}
    "Os objetivos específicos devem refletir subprodutos ou um detalhamento do objetivo principal. N ão se deve, a
    princípio, mencionar como objetivo específico passos que são meramente intermediários para atingir o objetivo
    geral."
    \lipsum[5] \cite{PESQUISA:RAUL}.
    \end{citacao}


\subsection{Justificativa}    

    O trabalho precisa possuir uma justificativa:
    \begin{citacao}
    "A justificativa deve se referir principalmente à hipótese de trabalho, ou seja, deve-se justificar a escolha de uma
    hipótese em vez de tentar justificar apenas a importância do tema da pesquisa. Usualmente a importância do
    tema da pesquisa já foi abordada na contextualização do problema. Então não há necessidade de repetir essa
    justificativa."
    \lipsum[5] \cite{PESQUISA:RAUL}.
    \end{citacao}


\section{Revisão da Literatura (ou Revisão Bibliográfica)}

Especial atenção ao que este capítulo deve conter:
    \begin{citacao}
    "Revisão bibliográfica, conforme já comentado, não produz conhecimento novo, mas apenas supre as
    deficiências de conhecimento que o pesquisador tem em uma determinada área. Portanto, ela deve ser muito
    bem planejada e conduzida.
    (...)
    Quando se faz uma pesquisa em que alguma técnica de computação é aplicada a alguma outra área do
    conhecimento, é necessário que se faça a revisão bibliográfica sobre a técnica em si, sobre a área de aplicação e,
    mais do que tudo, sobre as aplicações que já foram tentadas com essa técnica ou com técnicas semelhantes na
    mesma área ou em áreas equivalentes. Exemplificando, um aluno pretende desenvolver um sistema
    multiagentes para auxiliar controladores de voo. Esse aluno deve conhecer profundamente os sistemas
    multiagentes e deverá conhecer também os problemas que os controladores de voo enfrentam para exercer sua
    profissão. Porém, ele não deve pensar, como algumas vezes acontece, que essa é a primeira vez que alguém vai
    tentar desenvolver um sistema multiagentes para esse tipo de aplicação."
    \cite{PESQUISA:RAUL}.
    \end{citacao}

Toda a revisão da literatura deve ser basear primordialmente em livros e artigos científicos ranqueados Qualis CAPES. De forma geral, todo parágrafo deve conter AO MENOS uma citação bibliográfica.

% ---
\subsection{Assunto 1}
\lipsum[1]
\subsection{Assunto 2}
\lipsum[1]
\subsection{Assunto 3}
\lipsum[1]
\subsection{Assunto X}
\lipsum[1]


\section{Métodos de Pesquisa OU Materiais e métodos}

Segundo \citeonline{PESQUISA:DEMO}, metodologia significa, “na origem do termo, estudo dos caminhos, dos instrumentos usados para se fazer ciência”.

Completando a linha de raciocínio, o autor acrescenta:

    \begin{citacao}
    “Alguns entendem por pesquisa o trabalho de coletar dados, sistematizá-los e, a partir daí fazer uma descrição da real-dade. Outros, fixam-se no patamar teórico e entendem por pesquisa o estudo e a produção de quadros teóricos de referência que estaria na origem da explicação da realidade. Descrever restringe-se a constatar o que já existe. Explicar corresponde a desvendar por que existe. Outros mais acreditam que pesquisar inclui teoria e prática. Porque compreender a realidade e nela intervir formam um todo só, tornando-se vício oportunista ficar apenas na constatação descritiva ou apenas na especulação teórica.”
    \lipsum[5] \cite{PESQUISA:DEMO}.
    \end{citacao}



As seções a seguir são sugestões do que pode estar na metodologia. Conversem com o(s) professor(es) em busca de ajuda para definir quais as seções mais adequadas para cada trabalho.

\subsection{Tipo de Pesquisa}
\lipsum[1]

\subsection{Plano Amostral (se Pesquisa Quantitativa)}
\lipsum[1]

\subsection{Instrumento de Pesquisa e Escalas Utilizadas (Escalas se Pesquisa Quantitativa)}
\lipsum[1]

\subsection{Coleta de Dados}
\lipsum[1]

\subsection{Análise de Dados}
\lipsum[1]

\subsection{Materiais}
Para desenvolver uma aplicação web, faz-se necessário o uso de diversos materiais, os quais vão desde uma linguagem de programação específica até um navegador qualquer, dessa forma, serão listadas a seguir todas as ferramentas que serão utilizadas na elaboração do projeto.
	
 \subsection{Métodos}
Os métodos, modo como aplicamos as ferramentas no desenvolvimento, deixa claro como será feito todo o processo de criação do sistema.

\subsection{Embasamento Inicial}
\lipsum[1]

\subsection{Desenvolvimento do Software}
\lipsum[1]

\subsection{Metodologias de Desenvolvimento}
\lipsum[1]

\section{Desenvolvimento}

\subsection{Equipe}
\lipsum[1]

\subsection{Requisitos}
\lipsum[1]

\subsection{Modelagem}
\lipsum[1]

\subsection{Prototipagem}
\lipsum[1]

\section{POC}

A Prova de Conceito (\textit{Proof of Concept} (PoC)) que deve demonstrar a aderência das tecnologias escolhidas com a aplicação que deve ser desenvolvida. Essa prova de conceito deve demonstrar a comunicação desde o usuário até a base de dados e utilizar de forma simples as tecnologias escolhidas para demonstrar que
elas funcionam para o objetivo desejado.

\section{MVP}

O termo MVP foi popularizado por  \citeonline{ries2011lean}, onde ele descreve o conceito como segue:

"O MVP é o menor conjunto de recursos que permite que o empreendedor comece o processo de aprendizado com o mínimo de esforço e o máximo de aprendizado validado sobre os clientes."

Outro autor importante na área, \citeonline{blank2013startup}, define o MVP como:

"Uma ferramenta para testar hipóteses de negócios e iniciar o aprendizado, coletando o máximo de informações validadas sobre os clientes com o menor esforço possível."


% ---
% Finaliza a parte no bookmark do PDF, para que se inicie o bookmark na raiz
% ---
\bookmarksetup{startatroot}% 
% ---

% ---
% Conclusão
% ---
\section{Considerações finais}


De acordo com \citeonline{severino2016metodologia}, na seção de considerações finais o autor tem a oportunidade de fazer uma síntese dos principais pontos abordados e apresentar suas considerações finais sobre o assunto. Embora não haja uma estrutura fixa, existem algumas diretrizes comuns para escrever essa seção.

A seguir, algumas orientações gerais, para complementar a explicação:

1. Recapitule os principais pontos: Na seção de considerações finais, você pode revisitar os principais pontos discutidos ao longo do trabalho e resumir os resultados obtidos. É uma oportunidade para destacar a relevância do estudo e como ele contribui para o conhecimento existente.

2. Discuta as implicações dos resultados: Nessa seção, você pode discutir as implicações práticas e teóricas dos resultados do seu trabalho. 

3. Faça uma reflexão crítica: Use a seção de considerações finais para fazer uma reflexão crítica sobre as limitações do estudo e possíveis viéses. Discuta as dificuldades encontradas, bem como eventuais lacunas de conhecimento que podem ser exploradas por estudos futuros.

4. Encerre de forma concisa e impactante: Finalize a seção de considerações finais com uma frase ou parágrafo que resuma as principais conclusões e destaque a importância do estudo. É uma oportunidade para deixar uma impressão duradoura nos leitores.

Além do exposto acima, colocamos aqui uma outra possibilidade de estrutura para o documento:

\begin{enumerate}
\item Introdução
1.1. Objetivo
\item Concepção Inicial
\item Trabalhos Correlatos
    \begin{enumerate}
        \item Trabalho 1
        \item Trabalho 2
        \item Trabalho 3
        \item Trabalho X
    \end{enumerate}
\item Referencial Teórico
\item Materiais e métodos
\item Modelagem do Sistema
    \begin{enumerate}
        \item Diagrama de Casos de Uso
        \item Diagrama de Tabelas Relacionais
        \item Diagrama Entidade-Relacionamento
    \end{enumerate}
\item Funcionalidades
\item Considerações Finais
\end{enumerate}
Referências



% ----------------------------------------------------------
% ELEMENTOS PÓS-TEXTUAIS
% ----------------------------------------------------------
\postextual

% ----------------------------------------------------------
% Referências bibliográficas
% ----------------------------------------------------------
\bibliography{referencias}

% ----------------------------------------------------------
% Glossário
% ----------------------------------------------------------
%
% Há diversas soluções prontas para glossário em LaTeX. 
% Consulte o manual do abnTeX2 para obter sugestões.
%
%\glossary

% ----------------------------------------------------------
% Apêndices
% ----------------------------------------------------------

% ---
% Inicia os apêndices
% ---
\newpage
\begin{apendicesenv}

% ----------------------------------------------------------
\chapter{Nullam elementum urna vel imperdiet sodales elit ipsum pharetra ligula
ac pretium ante justo a nulla curabitur tristique arcu eu metus}
% ----------------------------------------------------------
Apêndices e anexos são materiais complementares ao texto que só devem ser incluídos quando forem imprescindíveis à compreensão deste.

Apêndices são textos elaborados pelo autor a fim de complementar sua argumentação.

Os apêndices devem aparecer após as referências, e os anexos, após os apêndices.

\end{apendicesenv}
% ---

% ----------------------------------------------------------
% Anexos
% ----------------------------------------------------------
\cftinserthook{toc}{AAA}
% ---
% Inicia os anexos
% ---
%\anexos
\newpage
\begin{anexosenv}

% ---
\chapter{Cras non urna sed feugiat cum sociis natoque penatibus et magnis dis
parturient montes nascetur ridiculus mus}
% ---

Anexos são os documentos não elaborados pelo autor, que servem de fundamentação, comprovação ou ilustração, como mapas, leis, estatutos etc.

Os apêndices devem aparecer após as referências, e os anexos, após os apêndices.

\end{anexosenv}

\end{document}
