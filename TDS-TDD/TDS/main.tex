\documentclass[12pt,a4paper,chapter=TITLE,section=TITLE,subsection=TITLE,subsubsection=TITLE]{article}
\usepackage{sbc-template}
\usepackage{times}			% Usa a fonte Times new Roman
\usepackage[T1]{fontenc}		% Selecao de codigos de fonte.
\usepackage[utf8]{inputenc}		% Codificacao do documento (conversão automática dos acentos)
\usepackage{indentfirst}		% Indenta o primeiro parágrafo de cada seção.
\usepackage{nomencl} 			% Lista de simbolos
\usepackage{color}				% Controle das cores
\usepackage{graphicx}			% Inclusão de gráficos
\usepackage{microtype} 
% Pacotes adicionais, usados apenas no âmbito do Modelo Canônico do abnteX2
% ---
\usepackage{lipsum}				% para geração de dummy text
% ---
		
% ---
% Pacotes de citações
% ---
\usepackage[brazilian,hyperpageref]{backref}	 % Paginas com as citações na bibl
\usepackage[alf]{abntex2cite}	% Citações padrão ABNT
% ---

% ---
% Configurações do pacote backref
% Usado sem a opção hyperpageref de backref
\renewcommand{\backrefpagesname}{Citado na(s) página(s):~}
% Texto padrão antes do número das páginas
\renewcommand{\backref}{}
% Define os textos da citação
\renewcommand*{\backrefalt}[4]{
	\ifcase #1 %
		Nenhuma citação no texto.%
	\or
		Citado na página #2.%
	\else
		%Citado #1 vezes nas páginas #2.%
        Citado nas páginas #2.%
	\fi}%

\title{Projeto CosCon\\ TDS }

\author{Guilherme Souza Malieni\inst{1} }


\address{Curso de Informática -- Instituto Federal de São Paulo
  (IFSP)\\
  \email{gui.souza.malieni@gmail.com}
}


% ---

% ---
% Configurações de aparência do PDF final

% alterando o aspecto da cor azul
\definecolor{blue}{RGB}{41,5,195}

% informações do PDF
\makeatletter
\hypersetup{
     	%pagebackref=true,
		pdftitle={\@title}, 
		pdfauthor={\@author},
    	pdfsubject={Modelo de artigo científico com abnTeX2},
	    pdfcreator={LaTeX with abnTeX2},
		pdfkeywords={abnt}{latex}{abntex}{abntex2}{atigo científico}, 
		colorlinks=true,       		% false: boxed links; true: colored links
    	linkcolor=blue,          	% color of internal links
    	citecolor=blue,        		% color of links to bibliography
    	filecolor=magenta,      		% color of file links
		urlcolor=blue,
		bookmarksdepth=4
		}
\makeatother
% --- 

% ---
% compila o indice
% ---
\makeindex
% ---

% ---
% Altera as margens padrões
% ---
\setlrmarginsandblock{3cm}{3cm}{*} %{ESQUERDA}{DIREITA}
\setulmarginsandblock{3.5cm}{2.5cm}{*} %{SUPERIOR}{INFERIOR}
\checkandfixthelayout
% ---

% --- 
% Espaçamentos entre linhas e parágrafos 
% --- 

% O tamanho do parágrafo é dado por (está definido dentro do sbc-template.sty):
%\setlength{\parindent}{1.3cm}

% Controle do espaçamento entre um parágrafo e outro (está definido dentro do sbc-template.sty):
%\setlength{\parskip}{0.2cm}  % tente também \onelineskip

% Espaçamento simples
\SingleSpacing


\begin{document}
  
  % Seleciona o idioma do documento (conforme pacotes do babel)
%\selectlanguage{english}
\selectlanguage {brazil}


% Retira espaço extra obsoleto entre as frases.
\frenchspacing 

% ----------------------------------------------------------
% ELEMENTOS PRÉ-TEXTUAIS
% ----------------------------------------------------------

%---
%
% Se desejar escrever o artigo em duas colunas, descomente a linha abaixo
% e a linha com o texto ``FIM DE ARTIGO EM DUAS COLUNAS''.
% \twocolumn[    		% INICIO DE ARTIGO EM DUAS COLUNAS
%
%---

% página de titulo principal (obrigatório)
\maketitle
Projeto CosCon-
Project CosCon

\begin{abstract}
  This meta-paper describes the style to be used in articles and short papers
  for SBC conferences. For papers in English, you should add just an abstract
  while for the papers in Portuguese, we also ask for an abstract in
  Portuguese (``resumo''). In both cases, abstracts should not have more than
  10 lines and must be in the first page of the paper.
\end{abstract}
     
\begin{resumo1} 
  Este meta-artigo descreve o estilo a ser usado na confecção de artigos e
  resumos de artigos para publicação nos anais das conferências organizadas
  pela SBC. É solicitada a escrita de resumo e abstract apenas para os artigos
  escritos em português. Artigos em inglês deverão apresentar apenas abstract.
  Nos dois casos, o autor deve tomar cuidado para que o resumo (e o abstract)
  não ultrapassem 10 linhas cada, sendo que ambos devem estar na primeira
  página do artigo.
\end{resumo1}



% ]  				% FIM DE ARTIGO EM DUAS COLUNAS
% ---

%\begin{center}\smaller
%\textbf{Data de submissão e aprovação}: elemento obrigatório. Indicar dia, mês e ano

%\textbf{Identificação e disponibilidade}: elemento opcional. Pode ser indicado o endereço eletrônico, DOI, suportes e outras informações relativas ao acesso.
%\end{center}

% ----------------------------------------------------------
% ELEMENTOS TEXTUAIS
% ----------------------------------------------------------
\textual

% ----------------------------------------------------------
% Introdução
% ----------------------------------------------------------
\section{Introdução}
O objetivo deste documento é expor as regras e formatos para o desenvolvimento dos documentos das disciplinas de TDS e PDS. Este utiliza como base tanto o modelo da SBC\footnote{\url{https://www.overleaf.com/latex/templates/sbc-conferences-template/blbxwjwzdngr}} quanto o fornecido pela abnTeX\footnote{\url{http://www.latex-project.org/lppl.txt}}.

O documento final deve ter \textbf{entre 12 e 15 páginas}, descontados na contagem apenas os anexos e apêndices. Cada professor é livre para definir junto aos alunos o que deve estar em cada uma das duas seções supracitadas.

As seções aqui colocadas são apenas EXEMPLOS do que deve estar contido, podendo ser alteradas em conjunto entre orientadores e alunos, assim como as devidas explicações dentro delas.

Atenção especial para a quantidade de texto colocada em cada uma das seções. De forma geral, não faz muito sentido colocar uma seção para escrever apenas um parágrafo ou 6, 7 linhas.

Para dicas e como fazer tabelas, listas, figuras, citações e etc, bem como evitar erros em projetos e no próprio documento, consultar (entre outros) os capítulos 6, 7, 8 e 9 do documento: \url{https://www.overleaf.com/project/58a3a66af9bb74023ba1bd56}.


Ao preparar a Introdução é importante estar atento a:
    \begin{citacao}
    "A introdução é a parte responsável pela apresentação do trabalho, desde a delimitação do tema até a forma como está organizado. Sugere-se que seja feita a introdução do assunto, de modo a: discorrer sobre o tema do trabalho; apresentação do problema e objetivos da pesquisa; exposição da justificativa; finalmente, informar em quantos capítulos o texto foi dividido e apresentar os principais elementos que os compõem . Por apresentar estes elementos, a introdução requer uma revisão detalhada ao final do trabalho para manter a coesão textual."  \cite{NormalizacaoIFSP}.
    \end{citacao}

\subsection{Objetivos}

Os objetivos precisam seguir uma linha de raciocínio ao serem traçados.
    \begin{citacao}
    "O objetivo normalmente comporta uma hipótese de trabalho. Um bom objetivo de pesquisa normalmente terá
    a forma “demonstrar que a hipótese x é verdadeira”.
    Nem todo objetivo pode ser considerado um bom objetivo de pesquisa. Por exemplo, algo do tipo “o objetivo
    deste trabalho é aumentar os meus conhecimentos na área de estudo” pode até ser muito sincero, mas não
    convence ninguém de que algum conhecimento novo para a humanidade será produzido. Portanto, isso deve ser evitado.
    Outro objetivo algumas vezes encontrado é a forma “propor…”. Alguma coisa é proposta, normalmente um
    método, uma abordagem, uma técnica, um algoritmo, uma comparação, ou qualquer outra coisa. A questão é:
    se o autor fizer a proposta, então o objetivo estará atingido? Se o aluno se propõe a propor e propôs, então está
    proposto! O que for proposto não é necessariamente melhor ou diferente daquilo que já existia antes. Então, o
    estágio da pesquisa neste caso ainda é dos mais ingênuos.
    É necessário que o objetivo diga que aquilo que está sendo proposto é melhor do que alguma outra coisa ou
    que resolve algum problema que antes não podia ser resolvido."  \cite{PESQUISA:RAUL}.
    \end{citacao}

Os objetivos PODEM ser divididos em: Principal e Específicos (ou Secundários).

Em relação ao objetivo principal:

    \begin{citacao}
    "Atenção especial é dada ao verbo que apresenta o objetivo.
    Analisa-se também se o objetivo apresentado define claramente uma pesquisa científica ou um objetivo
    tecnológico, como, por exemplo, a implementação de um sistema."
    \lipsum[5] \cite{PESQUISA:RAUL}.
    \end{citacao}

Em relação aos objetivos específicos (ou Secundários):

    \begin{citacao}
    "Os objetivos específicos devem refletir subprodutos ou um detalhamento do objetivo principal. N ão se deve, a
    princípio, mencionar como objetivo específico passos que são meramente intermediários para atingir o objetivo
    geral."
    \lipsum[5] \cite{PESQUISA:RAUL}.
    \end{citacao}


\subsection{Justificativa}    

    O trabalho precisa possuir uma justificativa:
    \begin{citacao}
    "A justificativa deve se referir principalmente à hipótese de trabalho, ou seja, deve-se justificar a escolha de uma
    hipótese em vez de tentar justificar apenas a importância do tema da pesquisa. Usualmente a importância do
    tema da pesquisa já foi abordada na contextualização do problema. Então não há necessidade de repetir essa
    justificativa."
    \lipsum[5] \cite{PESQUISA:RAUL}.
    \end{citacao}

\end{document}